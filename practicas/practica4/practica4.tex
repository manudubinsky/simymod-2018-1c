\documentclass[a4paper]{article}
\usepackage{a4wide}
\usepackage{amsmath, amscd, amssymb, amsthm, latexsym}
\usepackage[spanish,activeacute]{babel}
\usepackage{enumerate}
\usepackage{xspace}
\usepackage{longtable}
\usepackage{graphics}
\usepackage{listings}
\usepackage{hyperref}

\input{../../macros/Algo1Macros}

\begin{document}
\practica{4}{Modelos Físicos: Método de diferencias finitas}

\section{Introducción}

\subsection{Método de diferencias finitas}

Este método permite resolver sistemas de ecuaciones diferenciales 
(ordinarias y en derivadas parciales)

\bigskip

Vamos a emplear los siguientes libros de referencia:

\begin{enumerate}
	\item \emph{Finite Difference Methods for Ordinary and Partial 
	Differential Equations}
	\item \emph{Finite Difference Computing with Exponential Decay 
	Models}
	\item \emph{Finite Difference Computing with PDEs}
\end{enumerate}


\section{Ejercicios}

\subsection{Ejercicio 1}

Dada la función $u(x) = sin(x) + cos(x)$, realizar las siguientes 
actividades:

\begin{enumerate}
	\item Graficar $u(x)$ con matplotlib
	\item De acuerdo al ejemplo 1.1 del libro de referencia 1 (pag. 4), 
	realizar el análisis del error para la función $u(x) = sin(x) + 
	cos(x)$ para las discretizaciones $D_+u(x)$, $D_-u(x)$, $D_0u(x)$ en 
	$x = 1$
	\item Graficar en escala log-log los errores calculados en el punto 
	anterior (del libro 1, ver Figura 1.2, pag. 6).
\end{enumerate}

Referencia: \url{https://matplotlib.org/users/pyplot_tutorial.html}

\subsection{Ejercicio 2}
En base a los tres esquemas descriptos en la sección 1.1 del libro 2 
(\emph{Forward Euler}, \emph{Backward Euler}, \emph{Crank-Nicholson }), 
realizar discretizaciones para el ejericio 4.2 (pág. 112).

\subsection{Ejercicio 3}
En base a la implementación en la sección 1.1 del libro 2 (pág. 15):

\begin{enumerate}
	\item Realizar los puntos a), b) y c) del ejercicio 4.3 (pág 113)
	\item Realizar el ejercicio 4.4 (pág. 114)
\end{enumerate}
 
\subsection{Ejercicio 4}
Evaluar la posibilidad de realizar una animación de la Ley de 
Enfriamiento de Newton en la cual el color de una circunferencia varíe 
del rojo al azul a medida que se va enfriando.

\bigskip

Referencia 1: considerar el \emph{modelo\_simple2.py} de la práctica 3

\smallskip

Referencia 2: \url{https://matplotlib.org/1.5.1/api/colors_api.html}

\end{document}
