\documentclass[a4paper]{article}
\usepackage{a4wide}
\usepackage{amsmath, amscd, amssymb, amsthm, latexsym}
\usepackage[spanish,activeacute]{babel}
\usepackage{enumerate}
\usepackage{xspace}
\usepackage{longtable}
\usepackage{graphics}
\usepackage{listings}
\usepackage{hyperref}

\input{../../macros/Algo1Macros}

\begin{document}
\practica{5}{Modelos Físicos: Método de diferencias finitas 2}

\section{Introducción}

\subsection{Método de diferencias finitas}

Este método permite resolver sistemas de ecuaciones diferenciales 
(ordinarias y en derivadas parciales)

\bigskip

Vamos a emplear los siguientes libros de referencia:

\begin{enumerate}
	\item \emph{Finite Difference Methods for Ordinary and Partial 
	Differential Equations}
	\item \emph{Finite Difference Computing with Exponential Decay 
	Models}
	\item \emph{Finite Difference Computing with PDEs}
\end{enumerate}


\section{Ejercicios}

\subsection{Ejercicio 1}

De acuerdo con la descripción del libro \emph{Finite Difference 
Computing with PDEs}, se pide implementar la resolución de la 
ecuación de onda unidimensional por diferencias finitas.

\subsection{Ejercicio 2}
Implementar la posibilidad de visualizar animaciones de la cadena en 2D.

\bigskip

Referencia: basarse en las animaciones realizadas en las prácticas 
anteriores.

\subsection{Ejercicio 3}
Realizar experimentos (animaciones) variando la función ($I(x)$) y la 
velocidad de la onda ($c$).

\smallskip

Para cada una de las funciones que se detallan a continuación, 
considerar las velocidades ($c = 1, \frac{1}{2}, 2$) en el intervalo $x 
\in [0,10]$

\begin{itemize}
\item $I(x) = -\frac{1}{5} x^2 + 2 x$
\item $I(x) = 2 sin(x)$
\item \begin{equation}
  I(x) =
    \begin{cases}
      1 & \text{x $\in [0,1]$}\\
      0 & \text{en caso contrario}
    \end{cases}       
\end{equation}
\end{itemize}

\subsection{Ejercicio 4}
Realizar visualizaciones en 3D de la resolución de la ecuación de 
onda para los casos del punto 3.

\bigskip

Referencia: \emph{plot3D.py}.

 
\end{document}
