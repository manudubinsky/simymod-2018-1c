\documentclass[a4paper]{article}
\usepackage{a4wide}
\usepackage{amsmath, amscd, amssymb, amsthm, latexsym}
\usepackage[spanish,activeacute]{babel}
\usepackage{enumerate}
\usepackage{xspace}
\usepackage{longtable}
\usepackage{graphics}
\usepackage{listings}
\usepackage{hyperref}

\input{../../macros/Algo1Macros}

\begin{document}
\practica{6}{Distribuciones de Probabilidad}

\section{Introducción}

Las \emph{distribuciones de probabilidad} son funciones que devuelven 
la probabilidad de ocurrencia de los distintos eventos en un 
experimento. Por ejemplo, las probabilidades de cara y ceca en el 
lanzamiento de una moneda.

\bigskip

Hay \emph{distribuciones discretas} en el que los resultados del experimento 
pertenecen a un conjunto discreto (ej.: $\mathbb{N}$), por ejemplo el lanzamiento de una 
moneda o un dado. la cantidad de personas que hay en un negocio en 
momento dado.

\bigskip

Y hay \emph{distribuciones continuas} en el que los resultados del 
experimento toman valores en un rango continuo (ej.: $\mathbb{R}$), por ejemplo la 
temperatura de una habitación o el tiempo de llegada del próximo 
cliente a un negocio.

\bigskip

Cada \emph{variable aleatoria} se asocia a una distribución de 
probabilidades. 

\bigskip

Referencia general: \url{https://www.probabilitycourse.com}

\section{Ejercicios}

\subsection{Ejercicio 1}

La función \emph{nmupy.random.random} genera un valor  
aleatorio en el intervalo $[0,1)$. Se pide verificar mediante un 
conjunto de experimentos que los valores aleatorios corresponden a una 
\emph{distribución uniforme}:

\begin{enumerate}
\item Generar una lista de 100 valores y graficar un histograma con 50 
bins
\item Idem 1) con una lista de 1000 valores
\item Idem 1) con una lista de 10000 valores
\item Idem 1) con una lista de 100000 valores
\end{enumerate}

\bigskip

Referencia: 
\url{https://docs.scipy.org/doc/numpy-1.14.0/reference/generated/numpy.random.rand.html}

\smallskip

Referencia: 
\url{https://matplotlib.org/api/_as_gen/matplotlib.pyplot.hist.html}

\smallskip

Referencia: 
\url{https://en.wikipedia.org/wiki/Uniform_distribution_(continuous)}

\subsection{Ejercicio 2}

Escribir funciones para generar variables aleatorias asociadas a las 
distribuciones discretas detalladas en 
\url{https://www.probabilitycourse.com/chapter3/3_1_5_special_discrete_distr.php}

\begin{itemize}
\item $X \ \sim \ Bernoulli(p)$
\item $X \ \sim \ Geometric(p)$
\item $X \ \sim \ Binomial(n,p)$
\item $X \ \sim \ Pascal(m,p)$
\item $X \ \sim \ Hypergeometric(b,r,k)$
\end{itemize}

\subsection{Ejercicio 3}

Mediante el \emph{método inverso} escribir una función para 
generar una variable aleatoria con \emph{distribución exponencial} ($X 
\sim Exponential(\lambda)$).


\bigskip

Referencia: \url{https://en.wikipedia.org/wiki/Inverse_transform_sampling}

\smallskip

Referencia: 
\url{https://www.probabilitycourse.com/chapter4/4_2_2_exponential.php} 

\subsection{Ejercicio 4}

Utilizando la función del ejercicio 3 escribir otra función para 
generar una variable aleatoria con \emph{distribución Poisson} ($X 
\sim Poisson(\lambda)$).

\subsection{Ejercicio 5}

La \emph{distribución normal} no puede generarse mediante el método 
inverso (como la exponencial) porque no se puede calcular 
explícitamente su función acumulada. Por lo tanto, se requieren otros 
métodos para generarla. Mediante el \emph{método de Box-Muller} 
generar una distribución normal estándar.

\bigskip

Referencia (general): 
\url{https://www.probabilitycourse.com/chapter4/4_2_3_normal.php}

\smallskip

Referencia (Box-Muller): 
\url{https://en.wikipedia.org/wiki/Normal_distribution#Generating_values_from_normal_distribution}

\end{document}
