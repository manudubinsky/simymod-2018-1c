\documentclass[a4paper]{article}
\usepackage{a4wide}
\usepackage{amsmath, amscd, amssymb, amsthm, latexsym}
\usepackage[spanish,activeacute]{babel}
\usepackage{enumerate}
\usepackage{xspace}
\usepackage{longtable}
\usepackage{graphics}
\usepackage{listings}
\usepackage{hyperref}

\input{../../macros/Algo1Macros}

\begin{document}
\practica{1}{Modelo Abstracto: ``Juego de la Vida"}

\section{Introducción}

\subsection{Juego de la Vida}

El \textbf{Juego de la Vida} es un sistema que modela el ciclo de la vida 
de una población dispuesta en una grilla. Las reglas que 
determinan el nacimiento y la muerte de los individuos permiten 
considerar la evolución de la población. La historia, las reglas y el 
interés teórico que sucitó este juego  está detallado en \url{https://en.wikipedia.org/wiki/Conway%27s_Game_of_Life}

\subsection{ncurses}

La librería \textbf{ncurses} permite controlar una terminal en modo 
texto. La historia de la librería está descripta en 
\url{https://en.wikipedia.org/wiki/Ncurses}. Hay un \emph{wrapper} de 
la librería para python 
\url{https://docs.python.org/3/howto/curses.html}. La funcionalidad que 
vamos a requerir del wrapper en python es la siguiente:

\begin{itemize}
	\item Sección \textbf{Starting and ending a curses application}
	\item Otros comandos: \textbf{nodelay}, \textbf{getch}
\end{itemize}

\subsection{Matrices en python}

Para inicializar una matriz en python ver

\smallskip

\url{https://stackoverflow.com/questions/6667201/how-to-define-a-two-dimensional-array-in-python}

\section{Ejercicios}

\subsection{Ejercicio 1}

Implementar una versión reducida del \textbf{juego de la vida} en 
python con las siguientes características:

\begin{itemize}
	\item En lugar de considerar una grilla infinita, vamos a considerar  
	una grilla cuadrada de 25 $\times$ 25.
	\item La visualización de la evolución del sistema debe estar 
	implementada en modo texto (a las casillas que contienen un 
	individuo les corresponderá una ``X" \ en la pantalla y las que estén 
	vacías un <<espacio>> `` ").
	\item Para hacer pruebas el input de la configuración inicial puede estar 
	hardcodeado, pero en la versión final tiene que estar implementado 
	mediante un archivo de texto que indique en cada línea los pares 
	(<<fila>>, <<columna>>) en los que hay un individuo. 
	\item Al principio de la ejecución hay que mostrar la 
	configuración inicial, con la <<tecla>> ``b" (begin) comenzará a 
	ejecutarse la evolución del sistema; y con la <<tecla>> ``q" (quit), 
	se debe terminar el programa
	\item \textbf{Observación}: considerar la posibilidad de implementar el 
	modelo con una matriz que tenga una fila/columna vacía extra en cada 
	borde para evitar tener que considerar muchos casos particulares.	
\end{itemize}

\subsection{Ejercicio 2}
Visualizar y verificar el comportamiento de las configuraciones 
iniciales de tipo: ``Still lifes", ``Oscillators" \ y ``Spaceships", 
descriptos en la sección \emph{Examples of patterns} de \url{https://en.wikipedia.org/wiki/Conway%27s_Game_of_Life}

\end{document}
