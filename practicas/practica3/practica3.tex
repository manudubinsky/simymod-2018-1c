\documentclass[a4paper]{article}
\usepackage{a4wide}
\usepackage{amsmath, amscd, amssymb, amsthm, latexsym}
\usepackage[spanish,activeacute]{babel}
\usepackage{enumerate}
\usepackage{xspace}
\usepackage{longtable}
\usepackage{graphics}
\usepackage{listings}
\usepackage{hyperref}

\input{../../macros/Algo1Macros}

\begin{document}
\practica{3}{Modelos Físicos: Mecánica Clásica II}

\section{Introducción}

\subsection{Mecánica Clásica}

La \textbf{Mecánica Clásica} es una teoría que permite decribir el 
movimiento de objetos macroscópicos: desde proyectiles hasta el 
movimiento de los planetas y las galaxias. Provee resultados muy 
precisos siempre y cuando los objetos: 1) no sean demasiado pesados, 2) 
no sean demasido pequeños, 3) la veloacidad no esté cerca de la 
velocidad de la luz. Para tener una breve descripción de los conceptos 
de la teoría ver: \url{https://en.wikipedia.org/wiki/Classical_mechanics}

\bigskip

Como referencia para los ejercicios vamos a utilizar el libro \url{http://farside.ph.utexas.edu/teaching/301/lectures/}

\bigskip

La descripciòn teórica del movimiento circular uniforme se puede ver en \url{http://farside.ph.utexas.edu/teaching/301/lectures/node86.html}

\subsection{Modelos Simples}

Para realizar esta práctica se proveen dos modelos simples en python 
que utilizan la funcionalidad de animación del módulo 
\emph{matplotlib}:

\begin{itemize}
	\item \textbf{modelo\_simple.py}: es el mismo modelo de la práctica 2. 
	\item \textbf{modelo\_simple2.py}: es un modelo que que diseña un 
	circunferencia de forma más concreta. 
\end{itemize}


\section{Ejercicios}

\subsection{Ejercicio 1}

Realizar una simulación de una circunferencia de radio 1 que describa un 
movimiento circular uniforme de radio 10 en torno al origen.

\subsection{Ejercicio 2}

Realizar una simulación de una circunferencia de radio 1 que describa un 
movimiento circular uniforme de radio 5 en torno a otra circunferencia 
de radio 1 centrada en el (2,2).


\subsection{Ejercicio 3}

Realizar un modelo bidimensional a escala de la órbita lunar alrededor 
de la Tierra y de ambos alrededor del Sol. Considerar las siguientes 
simplificaciones del modelo:

\begin{itemize}
	\item Las órbitas son circulares de velocidad uniforme.
	\item Los tres cuerpos celestes pueden tener el mismo tamaño (esto 
	se debe a que el tamaño del Sol es mucho mayor que el de la 
	Tierra y eso impediría verlos a ambos en la simulación).
	\item Considerar que la luna realiza una órbita (en torno a la 
	Tierra) por mes y que la Tierra realiza una órbita (en torno al Sol) 
	en doce meses. 
	\item Considerar las escalas de las distiancias Luna-Tierra y 
	Tierra-Sol (o sea los radios de las órbitas).
	\item Ubicar al Sol en el centro de coordenadas (0,0)
\end{itemize}

\end{document}
