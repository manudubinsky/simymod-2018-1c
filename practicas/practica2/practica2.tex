\documentclass[a4paper]{article}
\usepackage{a4wide}
\usepackage{amsmath, amscd, amssymb, amsthm, latexsym}
\usepackage[spanish,activeacute]{babel}
\usepackage{enumerate}
\usepackage{xspace}
\usepackage{longtable}
\usepackage{graphics}
\usepackage{listings}
\usepackage{hyperref}

\input{../../macros/Algo1Macros}

\begin{document}
\practica{2}{Modelos Físicos: Mecánica Clásica}

\section{Introducción}

\subsection{Mecánica Clásica}

La \textbf{Mecánica Clásica} es una teoría que permite decribir el 
movimiento de objetos macroscópicos: desde proyectiles hasta el 
movimiento de los planetas y las galaxias. Provee resultados muy 
precisos siempre y cuando los objetos: 1) no sean demasiado pesados, 2) 
no sean demasido pequeños, 3) la veloacidad no esté cerca de la 
velocidad de la luz. Para tener una breve descripción de los conceptos 
de la teoría ver: \url{https://en.wikipedia.org/wiki/Classical_mechanics}

\bigskip

Como referencia para los ejercicios vamos a utilizar el libro \url{http://farside.ph.utexas.edu/teaching/301/lectures/}

\subsection{Modelo Simple}

Para realizar esta práctica se provee un modelo simple en python 
que utiliza la funcionalidad de animación del módulo 
\emph{matplotlib}. Una particula está modelada mediante un vector de 4 
componentes: $[x, y, v_x, v_y]$, donde el par $[x,y]$ corresponde a la 
posición y el par $[v_x, v_y]$ son las componentes del vector velocidad.

\section{Ejercicios}

\subsection{Ejercicio 1}

Realizar una simulación ideal (sin rozamiento) de un proceso de caída libre \url{http://farside.ph.utexas.edu/teaching/301/lectures/node19.html#e218}

\begin{itemize}
	\item ¿Cuánto tarda en caer un objeto desde 100 metros? Comparar el 
	resultado que arroja la simulación y el modelo de la teoría.
\end{itemize}

\subsection{Ejercicio 2}

Realizar una simulación del lanzamiento de una bala de cañón y 
verificar el ejemplo en \url{http://farside.ph.utexas.edu/teaching/301/lectures/node38.html}

\subsection{Ejercicio 3}

Realizar una simulación del modelo más simple de colisión 
unidimensional: entre dos partículas de la misma masa \url{http://farside.ph.utexas.edu/teaching/301/lectures/node76.html}

\subsection{Ejercicio 4}

De acuerdo al documento \emph{2dcollisions2.pdf}, realizar una 
simulación del modelo más simple de colisión bidimensional: entre 
dos partículas de la misma masa.

\begin{itemize}
	\item Modelar un conjunto de 50 partículas colisionando entre sí.
\end{itemize}

\end{document}
